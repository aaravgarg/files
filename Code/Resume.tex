\documentclass[letterpaper,11pt]{article}

\usepackage{latexsym}
\usepackage[empty]{fullpage}
\usepackage{titlesec}
\usepackage{marvosym}
\usepackage[usenames,dvipsnames]{color}
\usepackage{verbatim}
\usepackage{enumitem}
\usepackage[hidelinks]{hyperref}
\usepackage{fancyhdr}
\usepackage[english]{babel}
\usepackage{tabularx}
\input{glyphtounicode}


%----------FONT OPTIONS----------
% sans-serif
% \usepackage[sfdefault]{FiraSans}
% \usepackage[sfdefault]{roboto}
% \usepackage[sfdefault]{noto-sans}
% \usepackage[default]{sourcesanspro}

% serif
% \usepackage{CormorantGaramond}
% \usepackage{charter}


\pagestyle{fancy}
\fancyhf{} % clear all header and footer fields
\fancyfoot{}
\renewcommand{\headrulewidth}{0pt}
\renewcommand{\footrulewidth}{0pt}

% Adjust margins
\addtolength{\oddsidemargin}{-0.5in}
\addtolength{\evensidemargin}{-0.5in}
\addtolength{\textwidth}{1in}
\addtolength{\topmargin}{-.5in}
\addtolength{\textheight}{1.0in}

\urlstyle{same}

\raggedbottom
\raggedright
\setlength{\tabcolsep}{0in}

% Sections formatting
\titleformat{\section}{
  \vspace{-4pt}\scshape\raggedright\large
}{}{0em}{}[\color{black}\titlerule \vspace{-5pt}]

% Ensure that generate pdf is machine readable/ATS parsable
\pdfgentounicode=1

%-------------------------
% Custom commands
\newcommand{\resumeItem}[1]{
  \item\small{
    {#1 \vspace{-2pt}}
  }
}

\newcommand{\resumeSubheading}[4]{
  \vspace{-2pt}\item
    \begin{tabular*}{0.97\textwidth}[t]{l@{\extracolsep{\fill}}r}
      \textbf{#1} & #2 \\
      \textit{\small#3} & \textit{\small #4} \\
    \end{tabular*}\vspace{-7pt}
}

\newcommand{\resumeSubSubheading}[2]{
    \item
    \begin{tabular*}{0.97\textwidth}{l@{\extracolsep{\fill}}r}
      \textit{\small#1} & \textit{\small #2} \\
    \end{tabular*}\vspace{-7pt}
}

\newcommand{\resumeProjectHeading}[2]{
    \item
    \begin{tabular*}{0.97\textwidth}{l@{\extracolsep{\fill}}r}
      \small#1 & #2 \\
    \end{tabular*}\vspace{-7pt}
}

\newcommand{\resumeSubItem}[1]{\resumeItem{#1}\vspace{-4pt}}

\renewcommand\labelitemii{$\vcenter{\hbox{\tiny$\bullet$}}$}

\newcommand{\resumeSubHeadingListStart}{\begin{itemize}[leftmargin=0.15in, label={}]}
\newcommand{\resumeSubHeadingListEnd}{\end{itemize}}
\newcommand{\resumeItemListStart}{\begin{itemize}}
\newcommand{\resumeItemListEnd}{\end{itemize}\vspace{-5pt}}

%-------------------------------------------
%%%%%%  RESUME STARTS HERE  %%%%%%%%%%%%%%%%%%%%%%%%%%%%


\begin{document}

%----------HEADING----------
% \begin{tabular*}{\textwidth}{l@{\extracolsep{\fill}}r}
%   \textbf{\href{http://sourabhbajaj.com/}{\Large Sourabh Bajaj}} & Email : \href{mailto:sourabh@sourabhbajaj.com}{sourabh@sourabhbajaj.com}\\
%   \href{http://sourabhbajaj.com/}{http://www.sourabhbajaj.com} & Mobile : +1-123-456-7890 \\
% \end{tabular*}

\begin{center}
    \textbf{\Huge \scshape Aarav Garg} \\ \vspace{1pt}
    \small +1 (765)-543-8630 $|$ \href{mailto:gargaarav79@gmail.com}{\underline{gargaarav79@gmail.com}} $|$ 
    \href{https://linkedin.com/in/aaravgarg}{\underline{linkedin.com/in/aaravgarg}} $|$
    \href{https://github.com/aaravgarg}{\underline{github.com/aaravgarg}}
\end{center}


%-----------EDUCATION-----------
\section{Education}
  \resumeSubHeadingListStart
    \resumeSubheading
      {Purdue University}{West Lafayette, IN}
      {Bachelor of Science in Robotics Engineering (GPA 3.91)}{Expected Graduation - 2026}
      \resumeItemListStart
        \resumeItem{\textbf{Relevant Coursework}: C, Siemens NX, Electrical Systems, Industrial Robot Programming, Statics, Dynamics}
      \resumeItemListEnd
  \resumeSubHeadingListEnd


%-----------EXPERIENCE-----------
\section{Experience}
  \resumeSubHeadingListStart

  \resumeSubheading
      {Founding President}{March 2024 -- Present}
      {\href{https://humanoidrobot.club/}{Humanoid Robot Club Purdue}}{West Lafayette, IN}
      \resumeItemListStart
        \resumeItem{First group of students building a walking bipedal humanoid robot capable of space exploration.}
        \resumeItem{Raised \$150,000+ and 800+ members within 6 months of launch; now one of the largest Purdue tech orgs.}
        \resumeItem{Innovating bipedal motion by integrating a prosthetic ankle to improve adaptability on uneven terrains.}
        \resumeItem{Leading the integration of electrical systems and embedded software, including power distribution, IMU-based balance control, and actuator-driven movement, utilizing Jetson Nano.}
    \resumeItemListEnd

  \resumeSubheading
      {Undergraduate Research Assistant}{August 2024 -- Present}
      {MARS Lab [NSF-funded project under Dr. Yu She]}{West Lafayette, IN}
      \resumeItemListStart
        \resumeItem{Engineered underwater gliding robot for exploring cavities in icebergs–using Siemens NX.}
        \resumeItem{Integrated Arduino Mega with actuators and sensors; programmed the robot in C++.}
        \resumeItem{Designed and manufactured a custom Printed Circuit Board (PCB) to make the electrical system compact and streamline the overall assembly process.}
      \resumeItemListEnd

  \resumeSubheading
      {Technical Project Manager}{September 2023 -- August 2024}
      {Sphero Swarm (Purdue Robotics Club)}{West Lafayette, IN}
      \resumeItemListStart
        \resumeItem{Launched NSF-funded (\$30k) project to develop framework for controlling Sphero robots to simulate polymers.}
        \resumeItem{Led 30 team members across 5 technical subteams. Developed control algorithms for swarm control using ESP32.}
        \resumeItem{Optimized computer vision feedback loops to achieve precise movement with a layer of collision avoidance.}
    \resumeItemListEnd

    \resumeSubheading
      {Founder \& CEO}{January 2021 -- December 2023}
      {Tech Nuttiez}{Hyderabad, India}
      \resumeItemListStart
        \resumeItem{Ed-tech startup for robotics education. Led team of 30+ people. Built mobile app in Flutter + Firebase.}
        \resumeItem{1.2 Million+ students impacted and 15,000+ active users from 185+ countries.}
      \resumeItemListEnd
      
      
% -----------Multiple Positions Heading-----------
%    \resumeSubSubheading
%     {Software Engineer I}{Oct 2014 - Sep 2016}
%     \resumeItemListStart
%        \resumeItem{Apache Beam}
%          {Apache Beam is a unified model for defining both batch and streaming data-parallel processing pipelines}
%     \resumeItemListEnd
%    \resumeSubHeadingListEnd
%-------------------------------------------

    

    % \resumeSubheading
    %   {Artificial Intelligence Research Assistant}{May 2019 -- July 2019}
    %   {Southwestern University}{Georgetown, TX}
    %   \resumeItemListStart
    %     \resumeItem{Explored methods to generate video game dungeons based off of \emph{The Legend of Zelda}}
    %     \resumeItem{Developed a game in Java to test the generated dungeons}
    %     \resumeItem{Contributed 50K+ lines of code to an established codebase via Git}
    %     \resumeItem{Conducted  a human subject study to determine which video game dungeon generation technique is enjoyable}
    %     \resumeItem{Wrote an 8-page paper and gave multiple presentations on-campus}
    %     \resumeItem{Presented virtually to the World Conference on Computational Intelligence}
    %   \resumeItemListEnd

  \resumeSubHeadingListEnd


%-----------PROJECTS-----------
\section{Projects}
    \resumeSubHeadingListStart
      \resumeProjectHeading
          {\textbf{\href{https://www.instructables.com/Automated-Robotic-Arm-That-Learns-Ft-Tinkercad-Ard/}{\underline{Robotic Arm That Learns}}} $|$ \emph{Arduino Mega 2560, C++, KiCad, Siemens NX}}{}
          \resumeItemListStart
            \resumeItem{Programmed the arm with Arduino Mega 2560 using inverse kinematics for smooth movement.}
            \resumeItem{Designed custom PCB in KiCad that integrates potentiometers and servo motors for precise control and feedback.}
            \resumeItem{Engineered the arm and control panel in NX, incorporating the EEZYbotARM framework for stability.}
          \resumeItemListEnd
      \resumeProjectHeading
          {\textbf{\href{https://www.instructables.com/Spice-Box-That-Helps-You-Cook-Faster/}{\underline{Smart Spice Box}}} $|$ \emph{Arduino Uno, C++, KiCad, Siemens NX}}{}
          \resumeItemListStart
            \resumeItem{Developed a state-machine algorithm to control 999 unique states with 3 tactile buttons using an Arduino Uno.}
            \resumeItem{Utilized Siemens NX to design the enclosure, enhancing user ergonomics and compatibility with the electronics.}
            \resumeItem{Awarded \#1 prize in an international Autodesk Innovation Challenge.}
          \resumeItemListEnd
      \resumeProjectHeading
          {\textbf{\href{https://www.instructables.com/Pocket-Weather-Station-Your-Self-Care-Weather-Assi/}{\underline{Pocket Weather Station}}} $|$ \emph{Arduino Nano, C++, KiCad, Siemens NX}}{}
          \resumeItemListStart
            \resumeItem{Designed compact handheld device to measure real-time weather conditions using DHT11 and Arduino Nano.}
            \resumeItem{Modified design to use DHT22 sensor to increase meteorological reading accuracy from 97\% to 99.67\%.}
          \resumeItemListEnd
    \resumeSubHeadingListEnd



%
%-----------PROGRAMMING SKILLS-----------
\section{Technical Skills}
 \begin{itemize}[leftmargin=0.15in, label={}]
    \small{\item{
     \textbf{Programming}{: Java, Python, C/C++, Flutter, Firebase, SQL, MATLAB, Arduino, ROS, Verilog} \\
     \textbf{Platforms}{: Solidworks, Siemens NX, Mechanical CAD/CAM, Altium, KiCad, GitHub, Agile, Computer Architecture} \\
     \textbf{Hardware}{: System/Board (PCB) Design, Analog Circuit Design, Schematic Design, CAN, SPI, UART, I2C, Ultrasound, LiDAR, ECU Design, Structural Analysis, Testing, Power Systems, RF Circuit Design, ASIC, Camera Sensor, Thermal} \\
     \textbf{Algorithms}{: Motion Planning, Pathfinding, Collision Avoidance, Feedback Control Systems, SLAM, PID} \\
    }}
 \end{itemize}


%-------------------------------------------
\end{document}
